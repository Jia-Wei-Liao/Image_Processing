%設定頁面
\documentclass[12pt,a4paper]{article}
\usepackage[margin=1in,a4paper]{geometry}

%設定中文
\usepackage{xeCJK} 
\setCJKmainfont{標楷體} 
\XeTeXlinebreaklocale "zh"   
\XeTeXlinebreakskip = 0pt plus 1pt 

%浮水印
%\usepackage{draftwatermark}
%\SetWatermarkText{\bf NTNU MATH}
%\SetWatermarkScale{0.7}

%圖片
\usepackage{graphicx}
\usepackage{subfigure}

%頁首頁尾
\makeatother
\usepackage{fancyhdr}

%顏色
\usepackage{xcolor}

%表格顏色
\usepackage{colortbl}

%設定數學
\usepackage{amsmath, amsthm, amssymb}
\makeatletter

%自定圈圈標號
\usepackage{pstricks,pstricks-add}
\newcommand\textc[1]{{\begin{pspicture*}
(-0.25,-0.2)(0.25,0.3)\rput[c](0,0)
{\large \textcircled{\footnotesize #1}}
\end{pspicture*} }}

%自訂向量符號
\def\leftharpoonfill@{\arrowfill@\leftharpoonup\relbar\relbar}
\def\rightharpoonfill@{\arrowfill@\relbar\relbar\rightharpoonup}
\newcommand\rbjt{\mathpalette{\overarrow@\rightharpoonfill@}}
\newcommand\lbjt{\mathpalette{\overarrow@\leftharpoonfill@}}

%自訂定理
\newtheorem*{thm}{Theorem}
\newtheorem*{lem}{Lemma}
\newtheorem*{de}{Definition}
\newtheorem*{rmk}{Remark}
\newtheorem*{ex}{Example}
\newtheorem*{pf}{Proof}
\newtheorem*{sol}{Solution}

%程式碼
\usepackage{listings}
\usepackage{color}

\definecolor{dkgreen}{rgb}{0,0.6,0}
\definecolor{gray}{rgb}{0.5,0.5,0.5}
\definecolor{mauve}{rgb}{0.58,0,0.82}

\lstset{
  basicstyle={\small \ttfamily},
  frame=tb,
  language=Python,
  aboveskip=3mm,
  belowskip=3mm,
  showstringspaces=false,
  columns=flexible,
  basicstyle={\small\ttfamily},
  numbers=left,
  numbersep = 14pt,
  numberstyle=\tiny\color{gray},
  keywordstyle=\color{blue},
  commentstyle=\color{dkgreen},
  stringstyle=\color{mauve},
  breaklines=true,
  breakatwhitespace=true,
  tabsize=3,
  backgroundcolor=\color{gray!10}
}




%作者
\title{NTNU影像處理HW7,8}
\author{廖家緯}
\date{2020.4.29}

\begin{document}
\maketitle
%標題、作者、日期
\fontsize{12pt}{30pt}\selectfont
%設定字體大小、間距
\setlength{\baselineskip}{25pt}
%設定行距

\pagestyle{fancy}
\lhead{}
\chead{}
\rhead{}
\lfoot{}
\cfoot{\thepage}
\rfoot{}
\renewcommand{\headrulewidth}{0pt} %上線寬
\renewcommand{\footrulewidth}{0pt} %下線寬
%\renewcommand{\abstractname}{Executive Summary}




%正文開始
\begin{enumerate}
\item[1.]	
Show $f(ax) \Longleftrightarrow
\dfrac{1}{|a|}F\left(\dfrac{u}{a}\right)$
\,(Hint : let $y=ax$)\\
{\bf Solution:}
\begin{enumerate}
\item[($\Longrightarrow$)]
Let $F(u)=\mathcal{F}\{f(x)\}$.
WTS : $\mathcal{F}\{f(ax)\}
=\dfrac{1}{|a|}F\left(\dfrac{u}{a}\right)$.\\
Note that:
$F(u)= \displaystyle \frac{1}{2\pi}
\int_{-\infty}^{\infty}f(x) e^{-iux} dx$.
\begin{enumerate}
\item[Case 1:]
$a>0,$\\
Let $y=a x \Longrightarrow dy=adx$. Then
\begin{align*}
\displaystyle \mathcal{F}\{f(ax)\}
&=\frac{1}{2\pi} \int_{-\infty}^{\infty}
f(ax) e^{-iux} dx\\
&=\frac{1}{2\pi} \int_{-\infty}^{\infty}
f(y) e^{-iu\frac{y}{a}}\left(\frac{1}{a}\right) dy\\
&=\frac{1}{a}\left(
\frac{1}{2\pi} \int_{-\infty}^{\infty}
f(y) e^{-i\frac{u}{a}y}dy\right)\\
&=\frac{1}{a}F\left(\frac{u}{a}\right).
\end{align*}
\item[Case 2:]
$a<0,$\\
Let $y=a x \Longrightarrow dy=adx$. Then
\begin{align*}
\displaystyle \mathcal{F}\{f(ax)\}
&=\frac{1}{2\pi} \int_{-\infty}^{\infty}
f(ax) e^{-iux} dx\\
&=\frac{1}{2\pi} \int_{\infty}^{-\infty}
f(y) e^{-iu\frac{y}{a}}\left(\frac{1}{a}\right) dy\\
&=\frac{1}{-a}\left(
\frac{1}{2\pi} \int_{-\infty}^{\infty}
f(y) e^{-i\frac{u}{a}y}dy\right)\\
&=\frac{1}{-a}F\left(\frac{u}{a}\right).
\end{align*}
Hence $\mathcal{F}\{f(ax)\}
=\dfrac{1}{|a|}F\left(\dfrac{u}{a}\right)$
with $a \neq 0$.\\

\end{enumerate}
\item[$(\Longleftarrow)$]
Let $f(x)=\mathcal{F}^{-1}\{F(u)\}$.
WTS : $\mathcal{F}^{-1}
\{\frac{1}{|a|}F(\frac{u}{a})\}=f(ax)$.\\
Note that:
$f(x)= \displaystyle \int_{-\infty}^{\infty}
F(u)e^{iux}du$.

\begin{enumerate}
\item[Case 1:]
$a>0,$\\
Let $\dfrac{u}{a}=\omega
\Longrightarrow \dfrac{1}{a}du=d\omega$. Then
\begin{align*}
\mathcal{F}^{-1}
\left\{\frac{1}{|a|}F\left(\frac{u}{a}\right)\right\}
&=\displaystyle \int_{-\infty}^{\infty}\frac{1}{a}
F\left(\frac{u}{a}\right)e^{iux}du\\
&=\displaystyle \int_{-\infty}^{\infty}
\frac{1}{a} F(\omega)e^{i a\omega x}ad\omega\\
&=\displaystyle \int_{-\infty}^{\infty}
F(\omega)e^{i \omega (ax)}d\omega\\
&=f(ax)
\end{align*}

\item[Case 2:]
$a<0,$\\
Let $\dfrac{u}{a}=\omega
\Longrightarrow \dfrac{1}{a}du=d\omega$. Then
\begin{align*}
\mathcal{F}^{-1}
\left\{\frac{1}{|a|}F\left(\frac{u}{a}\right)\right\}
&=\displaystyle \int_{-\infty}^{\infty}\frac{1}{-a}
F\left(\frac{u}{a}\right)e^{iux}du\\
&=\displaystyle \int_{\infty}^{-\infty}
\frac{1}{-a} F(\omega)e^{i a\omega x}ad\omega\\
&=\displaystyle \int_{-\infty}^{\infty}
F(\omega)e^{i \omega (ax)}d\omega\\
&=f(ax).
\end{align*}
Hence $\mathcal{F}^{-1}
\{\frac{1}{|a|}F(\frac{u}{a})\}=f(ax)$
with $a \neq 0$.\\
\end{enumerate}

\end{enumerate}

\newpage
\item[2.]
The FFT presented in the class is called the
{\bf successive doubling method (SDM)}.  This 
method assumes that the sizes of images were
power of two. However, it is common that the 
size of an image is not power of two? Address 
the idea how do you solve this problem using 
the SDM.\\
{\bf Solution:}\\
根據之前的經驗,不足就補0。這題可以也可以試看看在邊界補0至2的次方。
\end{enumerate}










\end{document}