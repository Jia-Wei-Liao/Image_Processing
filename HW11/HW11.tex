%設定頁面
\documentclass[12pt,a4paper]{article}
\usepackage[margin=1in,a4paper]{geometry}

%設定中文
\usepackage{xeCJK} 
\setCJKmainfont{標楷體} 
\XeTeXlinebreaklocale "zh"   
\XeTeXlinebreakskip = 0pt plus 1pt 

%浮水印
%\usepackage{draftwatermark}
%\SetWatermarkText{\bf NTNU MATH}
%\SetWatermarkScale{0.7}

%圖片
\usepackage{graphicx}
\usepackage{subfigure}

%頁首頁尾
\makeatother
\usepackage{fancyhdr}

%顏色
\usepackage{xcolor}

%表格顏色
\usepackage{colortbl}

%設定數學
\usepackage{amsmath, amsthm, amssymb}
\makeatletter

%自定圈圈標號
\usepackage{pstricks,pstricks-add}
\newcommand\textc[1]{{\begin{pspicture*}
(-0.25,-0.2)(0.25,0.3)\rput[c](0,0)
{\large \textcircled{\footnotesize #1}}
\end{pspicture*} }}

%自訂向量符號
\def\leftharpoonfill@{\arrowfill@\leftharpoonup\relbar\relbar}
\def\rightharpoonfill@{\arrowfill@\relbar\relbar\rightharpoonup}
\newcommand\rbjt{\mathpalette{\overarrow@\rightharpoonfill@}}
\newcommand\lbjt{\mathpalette{\overarrow@\leftharpoonfill@}}

%自訂定理
\newtheorem*{thm}{Theorem}
\newtheorem*{lem}{Lemma}
\newtheorem*{de}{Definition}
\newtheorem*{rmk}{Remark}
\newtheorem*{ex}{Example}
\newtheorem*{pf}{Proof}
\newtheorem*{sol}{Solution}

%程式碼
\usepackage{listings}
\usepackage{color}

\definecolor{dkgreen}{rgb}{0,0.6,0}
\definecolor{gray}{rgb}{0.5,0.5,0.5}
\definecolor{mauve}{rgb}{0.58,0,0.82}

\lstset{
  basicstyle={\small \ttfamily},
  frame=tb,
  language=Python,
  aboveskip=3mm,
  belowskip=3mm,
  showstringspaces=false,
  columns=flexible,
  basicstyle={\small\ttfamily},
  numbers=left,
  numbersep = 14pt,
  numberstyle=\tiny\color{gray},
  keywordstyle=\color{blue},
  commentstyle=\color{dkgreen},
  stringstyle=\color{mauve},
  breaklines=true,
  breakatwhitespace=true,
  tabsize=3,
  backgroundcolor=\color{gray!10}
}




%作者
\title{NTNU影像處理HW11}
\author{廖家緯}
\date{2020.5.27}

\begin{document}
\maketitle
%標題、作者、日期
\fontsize{12pt}{30pt}\selectfont
%設定字體大小、間距
\setlength{\baselineskip}{25pt}
%設定行距

\pagestyle{fancy}
\lhead{}
\chead{}
\rhead{}
\lfoot{}
\cfoot{\thepage}
\rfoot{}
\renewcommand{\headrulewidth}{0pt} %上線寬
\renewcommand{\footrulewidth}{0pt} %下線寬
%\renewcommand{\abstractname}{Executive Summary}




%正文開始
\begin{enumerate}
\item[1.]	
Prove $\overline{A \oplus B}
=\overline{A} \Theta \widehat{B}$.\\
{\bf Proof:}
\begin{align*}
\overline{A \oplus B}
&=\overline{\{x|
(\widehat{B})_x \cap A \neq \varnothing \}}
\,\text{(by definition)}\\
&=\{x|(\widehat{B})_x \cap A=\varnothing \}\\
&=\{x|(\widehat{B})_x \subseteq \overline{A} \}\\
&=\overline{A} \Theta \widehat{B}
\,\text{(by definition)}\\ 
\end{align*}


\item[2.]	
Prove $\overline{A \circ B}
=\overline{A} \bullet \widehat{B}$.\\
{\bf Proof:}
\begin{align*}
\overline{A \circ B}
&=\overline{(A \Theta B) \oplus B}
\,\text{(by definition)}\\ 
&=\overline{(A \Theta B)} \Theta \widehat{B}
\, \text{(by 1.)}\\
&=(\overline{A} \oplus \widehat{B})
\Theta \widehat{B}\, \text{(by property)}\\
&=\overline{A} \bullet \widehat{B}
\,\text{(by definition)}\\ 
\end{align*}

\newpage
\item[3.]	
Prove if $A \subseteq C$, then
$({A \circ B}) \subseteq (C \circ B)$.\\
{\bf Proof:}\\
Let $x \in (A \circ B)$.
By definition, $x \in (A \Theta B) \oplus B$.\\
Then $x=p+q$
for some $p \in (A \Theta B)$ and
$q \in B$\\
Note that\\
$\because p \in (A \Theta B) \quad$
$\therefore$ by definition,
$p \in A$ and $B_p \subseteq A$\\
Moreover, $\because A \subseteq C \quad$
$\therefore p \in C$ and $B_p \subseteq C$
$\Longrightarrow p \in (C \Theta B)$\\
Hence $x=p+q$
for some $p \in (C \Theta B)$ and $q \in B$\\
$\Longrightarrow x \in (C \Theta B) \oplus B$
\,(by definition)\\
$\Longrightarrow x \in (C \circ B)$
\,(by definition)\\ 
Therefore, $({A \circ B}) \subseteq (C \circ B)$.
\item[4.]	
Prove if $A \subseteq C$, then
$({A \bullet B}) \subseteq ({C \bullet B})$.

\begin{tabular}{|l|}
\hline
{\bf Lemma.}\\
If $A \subseteq C$, then
$({A \oplus B}) \subseteq ({C \oplus B})$.\\
{\bf Proof:}\\
Let $x \in ({A \oplus B})$.
Then $x=p+q$ for some $p \in A$ and $q \in B$\\
$\because A \subseteq C \quad
\therefore p \in C$ and $q \in B$
$\quad \Longrightarrow x=p+q$
for some $p \in C$ and $q \in B$.\\
Hence $x \in (C \oplus B)$.
Therefore,
$({A \oplus B}) \subseteq ({C \oplus B})$\\
\hline
\end{tabular}

{\bf Proof:}\\
Let $x \in (A \bullet B)$.
By definition, $x \in (A \oplus B) \Theta B$.\\
Then $x \in (A \oplus B)$ and $B_x \subseteq (A \oplus B)$\\
Moreover, $\because A \subseteq C$
$\quad \therefore$ by Lemma,
$x \in (C \oplus B)$ and $B_x \subseteq (C \oplus B)$\\
$\Longrightarrow x \in (C \oplus B) \Theta B$.
\,(by definition)\\
$\Longrightarrow x \in (C \bullet B)$
\,(by definition)\\
Therefore, $({A \bullet B}) \subseteq ({C \bullet B})$.
\end{enumerate}











\end{document}